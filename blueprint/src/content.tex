% In this file you should put the actual content of the blueprint.
% It will be used both by the web and the print version.
% It should *not* include the \begin{document}
%
% If you want to split the blueprint content into several files then
% the current file can be a simple sequence of \input. Otherwise It
% can start with a \section or \chapter for instance.


\section{The Gauss-Lucas Theorem}

\begin{lemma}[Logarithmic Derivative of a Complex Polynomial]
  \label{lem:logDeriv_Polynomial}
  \lean{logDeriv_Polynomial}
  \leanok
  The logarithmic derivative of a complex polynomial
  is the sum of the inverses of its factors.
\end{lemma}
  
\begin{proof}
  TODO
\end{proof}

\begin{lemma}[Sum of inverses of polynomial factors on a root]
  \label{lem:sum_inv_sub_roots_eq_zero}
  \lean{sum_inv_sub_roots_eq_zero}
  \leanok
  If z is a root of the derivative of a polynomial p, but not a root of p itself, then the sum of the inverses of the differences between x and each root of p is zero.
\end{lemma}
  
\begin{proof}
  \uses{lem:logDeriv_Polynomial}
  TODO
\end{proof}

\begin{lemma}[The derivative root is a weighted average of roots]
  \label{lem:deriv_root_as_weighted_average_of_roots}
  \lean{deriv_root_as_weighted_average_of_roots}
  \leanok
  If z is a root of the derivative of a polynomial p, but not a root of p itself,
  then it can be expressed as a weighted average of the roots of p.
\end{lemma}
  
\begin{proof}
  \uses{lem:sum_inv_sub_roots_eq_zero}
  TODO
\end{proof}

\begin{theorem}[Gauss-Lucas]
  \label{thm:gauss_lucas}
  \lean{gauss_lucas}
  \leanok
  If P is a polynomial with complex coefficients, all roots of P' belong to the convex hull of the set of zeros of P.
\end{theorem}
  
\begin{proof}
  \uses{lem:deriv_root_as_weighted_average_of_roots}
  TODO
\end{proof}
